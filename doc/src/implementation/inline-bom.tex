\documentclass[11pt]{article}

\input{../common/common-defs}
\usepackage{graphicx}
\usepackage{../common/code}

\title{Manticore Implementation Note \\ Inline BOM}
\author{The Manticore Group}
\date{Draft of \today}

\begin{document}
\maketitle

\section{Overview}
The implementation of inline BOM is spread across the compiler. We list the most relevant files below.
\begin{itemize}
\item \texttt{BOMBoundVariableCheck} This module checks that variables in inline BOM programs are 
bound properly. In the process, we convert qualified identifiers to stamps.
\item \texttt{TranslatePrim} This module converts inline BOM parse trees to actual BOM code.
\end{itemize}

\section{Accessing inline BOM in the compiler}
Often, compiler passes elaborate some syntax into more basic operations. For instance, we elaborate the \texttt{pval} binding into operations over futures. To access these operations, the compiler pass needs a handle on the futures library code. This note describes how we obtain such a handle.

The two relevant operations, given below, are part of the translation environment (\texttt{TranslateEnv}). The first operation takes a qualified identifier, \eg{} \texttt{[``Future'', ``future'']}, and looks for \texttt{typedef}-bound type in the \emph{basis environment}. Similarly, the second operation takes a qualified identifier and looks for the associated HLOp in the basis environment. For an example of how the compiler uses these operations, see the \texttt{TranslatePValCilk5} module.

\begin{centercode}
val findBOMTyByPath : string list -> BOMTy.ty
val findBOMHLOpByPath : string list -> BOM.hlop
\end{centercode}

\end{document}
