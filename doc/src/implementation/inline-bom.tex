\documentclass[11pt]{article}

\input{../common/common-defs}
\usepackage{graphicx}
\usepackage{../common/code}

\title{Manticore Implementation Note \\ Inline BOM}
\author{The Manticore Group}
\date{Draft of \today}

\begin{document}
\maketitle

\section{Overview}
The implementation of inline BOM is spread across the compiler. The relevant files are below.
\begin{itemize}
\item \texttt{src/tools/mc/binding/bound-variable/bound-variable-check.sml} Variable binding for Manticore programs, which includes hooks for inline BOM.
\item \texttt{src/tools/mc/binding/bound-variable/bom-bound-variable-check.sml} Variable binding for inline BOM programs.
\item \texttt{src/tools/mc/translate/translate-prim.sml} Translation from inline BOM to BOM.
\end{itemize}

\section{Loading HLOps at compile time}
In many cases, we wish to create PML syntax that expands into several HLOps. For example, the pval binding form expands into a few HLOps from the single-toucher-future module. But because the compiler is responsible for doing this expansion, we do not not have a static handle on the HLOps. Instead we use a dynamic lookup at compile time. The following function in the HLOpEnv module does the job:
\begin{centercode}
  (* locate a HLOp by path name, e.g., Future1.@touch *)
    val findDefByPath : string list -> hlop_def option
\end{centercode}
\subsubsection{Example}
To look up the Future1 touch operation, we say:
\begin{centercode}
  case HLOpEnv.findDefByPath ["Future1", "touch"]
    of SOME def => (* do something with the hlop *)
     | NONE => raise Fail ``error: cannot find future1 touch''
\end{centercode}

\end{document}
